\documentclass[letter]{article}
\usepackage{amsmath}
\usepackage{amsfonts}
\usepackage{amssymb}
\usepackage{ifthen}
\usepackage{fancyhdr}
\usepackage{enumitem}

%%%
% Set up the margins to use a fairly large area of the page
%%%
\oddsidemargin=.2in
\evensidemargin=.2in
\textwidth=6in
\topmargin=0in
\textheight=9.0in
\parskip=.07in
\parindent=0in
\pagestyle{fancy}

%%%
% Set up the header
%%%
\newcommand{\setheader}[6]{
	\lhead{{\sc #1}\\{\sc #2} ({\small \it \today})}
	\rhead{
		{\bf #3} 
		\ifthenelse{\equal{#4}{}}{}{(#4)}\\
		{\bf #5} 
		\ifthenelse{\equal{#6}{}}{}{(#6)}%
	}
}

%%%
% Set up some shortcut commands
%%%
\newcommand{\R}{\mathbb{R}}
\newcommand{\C}{\mathbb{C}}
\newcommand{\N}{\mathbb{N}}
\newcommand{\Z}{\mathbb{Z}}
\newcommand{\Proj}{\mathrm{proj}}
\newcommand{\Perp}{\mathrm{perp}}
\newcommand{\proj}{\mathrm{proj}}
\newcommand{\Span}{\mathrm{span}}
\newcommand{\Null}{\mathrm{null}}
\newcommand{\Rank}{\mathrm{rank}}
\newcommand{\mat}[1]{\begin{bmatrix}#1\end{bmatrix}}
\renewcommand{\d}{\mathrm{d}}

%%%
% This is where the body of the document goes
%%%
\begin{document}
	\setheader{Math 281-2}{Homework 3}{Due Thursday, February 25}{}{}{}
	\begin{enumerate}
		\item Consider the second order linear homogeneous differential equation
			\begin{equation}
				\label{EQ}
				y''+y'-6y=0.
			\end{equation}
			\begin{enumerate}
				\item Write down the corresponding system of first order differential equations
					in matrix form.  That is, you system should be written as 
					\begin{equation}
						\label{EQ2}
						\mat{a'\\b'}=M\mat{a\\b}
					\end{equation}
					for some matrix $M$.
				\item Find all eigenvalues of $M$.
				\item An eigenvector for \eqref{EQ} is a solution $\phi$ such that $\phi'=k\phi$ for some $\phi$.
					Use your knowledge of how to solve \eqref{EQ} to find two eigenvectors for \eqref{EQ}.
				\item Use your eigenvectors from part (c) to find two eigenvectors
					for $M$.  That is, find vectors $\vec v$ and $\vec w$ so that $M\vec v= k_1\vec v$ and
					$M\vec w=k_2\vec w$ for scalars $k_1$ and $k_2$.
				\item Write down a flow for \eqref{EQ2} that passes through the point $(a,b) = (a_0,b_0)$ at 
					time $t=0$.
				\item Show that if $\vec \varphi_1:\R\to\R^2$ and $\vec\varphi_2:\R\to\R^2$ are flows for \eqref{EQ2},
					then so is any linear combination of $\vec\varphi_1$ and $\vec\varphi_2$.  (You may
					use facts about the linearity of matrix multiplication that you proved
					last homework).
			\end{enumerate}
		\item Consider the second order linear homogeneous differential equation
			\begin{equation}
				\label{EQgen}
				y''+\alpha y'+\beta y=0
			\end{equation}
			where $\alpha$ and $\beta$ are constants.
			\begin{enumerate}
				\item Write the corresponding system of first order linear differential equations in
					matrix form.
				\item Classify the $\alpha,\beta$ that give rise to sources, sinks, spiral sources,
					spiral sinks, saddles, and swirls.  Are there any $\alpha,\beta\in\R$ that don't
					fit into this classification?
				\item Classify the $\alpha,\beta$ that give rise to forward-stable, backward-stable,
					bi-stable, and unstable solutions to \eqref{EQgen}.
			\end{enumerate}

		\item The motion of a simple pendulum is governed by the equation
			\[
				\theta'' + \frac{g}{\ell}\sin\theta=0
			\]
			where $g$ is the acceleration due to gravity, $\ell$ is the length of
			the rigid pendulum arm, and $\theta$ is the angular displacement from vertical.  Since
			we're doing math in this course, we may assume $g=\ell=1$.
			\begin{enumerate}
				\item Write a system of first-order differential equations corresponding
					to the equation for a pendulum and draw a phase portrait for this system.
				\item Analyze the critical points for the pendulum system (by
					linearizing the system near the critical points).  Does their stability
					or instability make physical sense?  Explain.
				\item There are three fundamental behaviors for a simple pendulum: stationary,
					swinging back and forth, and spinning all the way around.  Draw, and clearly
					label, a flow corresponding to each behavior on your phase portrait.
				\item Given an initial condition $(t,\theta,\theta')=(0,\theta_0,0)$, if $\theta_0$ is
					small, a pendulum's motion is modeled well by simple harmonic oscillation (i.e.,
					a $\theta=k\cos(\omega t)$ for some $k,\omega$).  We're going to see exactly how well.

					Let $(a',b')=\vec F(a,b)$ be the system of ODEs corresponding to a simple
					pendulum, and let $(a',b')=\vec L(a,b)$ be the linearization of $\vec F$ around
					the critical point $(0,0)$.					
					Let $\vec\varphi_{\theta_0}$ be the
					flow along $\vec F$ with initial conditions
					$(t,\theta,\theta')=(0,\theta_0,0)$ and let $\vec\gamma_{\theta_0}$ be 
					the flow long $\vec L$ with initial conditions
					$(t,\theta,\theta')=(0,\theta_0,0)$.  Use numerical techniques to
					estimate the period of $\vec\varphi_{\theta_0}$ and $\vec\gamma_{\theta_0}$ 
					for at least three choices of $\theta_0$.  Give a range of $\theta_0$
					where the relative error between the period of $\vec\varphi_{\theta_0}$ and $\vec\gamma_{\theta_0}$
					is less than 5\%.  (Recall, relative error between quantities $a$ and $b$ is $|(a-b)/a|$.)

			\end{enumerate}

		\item Consider the vector field $\vec F(x,y) = (x+y^2,-y)$.
			\begin{enumerate}
				\item For each of the following functions, show whether or
					not it is a flow for $\vec F$.
					\begin{enumerate}
						\item $\vec \varphi(t) = (-e^{-2t},\sqrt{3}e^{-t})$
						\item $\vec \varphi(t) = (-e^{-4t},\sqrt{3}e^{-2t})$
						\item $\vec \varphi(t) = (\frac{4}{3}e^t-\frac{1}{3}e^{-2t},e^{-t})$
						\item $\vec \varphi(t) = (e^t-e^{-2t},e^{-t})$
					\end{enumerate}
				\item A \emph{constant flow} is a flow $\vec\varphi$ such that $\vec\varphi(t_1)=\vec \varphi(t_2)$ for
					all $t_1,t_2\in \R$.  Find all constant flows for $\vec F$. (Hint, think about $\vec\varphi\,'$ in
					this situation.)
				\item Classify all critical points of the system of differential equations given by
					$(a'(t),b'(t)) = \vec F(a(t),b(t))$.
				\item Find \emph{all} flows of $\vec F$ that pass through the point $(1,0)$.  Remember,
					if $\vec\varphi$ is a flow with this property, it is not a requirement that $\vec\varphi(0)=(1,0)$,
					only that $\vec \varphi(t_0)=(1,0)$ for some $t_0$.
				\item We call the flows $\vec\varphi_1$
					and $\vec \varphi_2$ time shifts of each other if $\vec \varphi_1(t)=\vec \varphi_2(t+t_0)$ for
					some $t_0$. Show that the flows from part (d) are time shifts of each other.  
			\end{enumerate}
		\item {\sc I hope we haven't forgotten about vector calculus}!
			Consider the vector field $\vec F:\R^2\to\R^2$ and imagine it describes the velocity
			of some fluid spread out on a plane (maybe we'll call this whole setup a laminar flow).

			Let $\vec\varphi_{\vec x}$ be the flow along $\vec F$ with initial conditions
			$\vec\varphi_{\vec x}(0)=\vec x$.  That is, $\vec\varphi_{\vec x}(t)$ 
			describes the position of a particle that starts at $\vec x$ at time zero
			and flows for $t$ seconds.

			Viewing $\vec\varphi_{\vec x}$ as a function of $\vec x$, we can view $\vec\varphi_{\vec x}(t)$
			as inducing a different coordinate change for each $t$.  Specifically, for a fixed $t_0$,
			we could think of the coordinate change 
			\[
				\vec x\mapsto \vec \varphi_{\vec x}(t_0).
			\]
			Let $c_{t_0}:\R^2\to\R^2$ be this coordinate change.

			Our goal will be to compute the volume form for this coordinate change and learn some fluid mechanics
			along the way!
			\begin{enumerate}
				\item Suppose the volume form for the change of coordinates induced by $c_t$ was $V_t(x,y)=3^{t}$.
					If you released a drop of die into the laminar flow governed by $\vec F$ and
					at time $0$ it had area $4$, what would the area be after 1 second?
				\item We can compute the volume form by estimating the change in volume of a tiny square
					under the function $c_{t_0}$ and taking the limit as the square gets infinitely
					tiny (just like we did before).  Let $R_{\varepsilon}^{(a,b)}$ be the square
					of side lengths $\varepsilon$ and lower left corner at $(a,b)$.  Compute
					\[
						V_{t_0}(a,b)=\lim_{\varepsilon\to 0}\frac{\text{area of }c_{t_0}(R_{\varepsilon}^{(a,b)})}{
						\text{area of }R_{\varepsilon}^{(a,b)}}.
					\]
					You may express your answer in terms of partial derivatives.  You may also
					find it helpful to assume that if $\varepsilon$ is tiny, then
					$c_{t_0}(R_{\varepsilon}^{(a,b)})$ is a parallelogram.  Further, recall that determinants
					can be used to compute the area of a parallelogram and that the determinant is
					a continuous function, and so you may move limits in and out of a determinant.
				\item The change of coordinates function $c_t$ has some fantastic properties.  Explain
					why $c_{t_1}\circ c_{t_2} = c_{t_1+t_2}$.  Using this fact, explain why 
					$V_{t_1+t_2}(x,y)=V_{t_1}(x,y)V_{t_2}(c_{t_1}(x,y))$.
				\item Write down the limit definition of the derivative of $\displaystyle\frac{\d V_t}{\d t}(x,y)$ at $t=t_0$
					and at $t=0$.
					Then, use your knowledge from the previous part to write down an expression for
					\[
						\left. \frac{\d V_t}{\d t}(x,y)\right|_{t=t_0}
					\]
					in terms of $\left. \frac{\d V_t}{\d t}\right|_{t=0}$.
				\item Use $\vec F$ to write down a linear approximation to $c_{t}$ when $t$ is very close
					to zero. Using this approximation, compute $\displaystyle \frac{\d V_t}{\d t}$ at $t=0$
					and
					show that
					\[
						\left. \frac{\d V_t}{\d t}\right|_{t=0}=\nabla \cdot \vec F.
					\]
				\item A flow is called \emph{incompressible} if $\nabla \cdot \vec F=0$.  Explain why.
					In particular, your explanation should in some way combine knowledge from
					parts (d) and (e) to draw strong conclusions about the volume form.
				\item Let $F(x,y) = (x+y^2,-y)$ from the previous problem.
					If you released a drop of die into the laminar flow governed by $\vec F$ and
					at time $0$ it had area $4$, what would the area be after 1 second?

			\end{enumerate}
	\end{enumerate}

\end{document}
