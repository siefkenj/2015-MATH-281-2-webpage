\documentclass[letter]{article}
\usepackage{amsmath}
\usepackage{amsfonts}
\usepackage{amssymb}
\usepackage{ifthen}
\usepackage{fancyhdr}
\usepackage{enumitem}

%%%
% Set up the margins to use a fairly large area of the page
%%%
\oddsidemargin=.2in
\evensidemargin=.2in
\textwidth=6in
\topmargin=0in
\textheight=9.0in
\parskip=.07in
\parindent=0in
\pagestyle{fancy}

%%%
% Set up the header
%%%
\newcommand{\setheader}[6]{
	\lhead{{\sc #1}\\{\sc #2} ({\small \it \today})}
	\rhead{
		{\bf #3} 
		\ifthenelse{\equal{#4}{}}{}{(#4)}\\
		{\bf #5} 
		\ifthenelse{\equal{#6}{}}{}{(#6)}%
	}
}

%%%
% Set up some shortcut commands
%%%
\newcommand{\R}{\mathbb{R}}
\newcommand{\N}{\mathbb{N}}
\newcommand{\Z}{\mathbb{Z}}
\newcommand{\Proj}{\mathrm{proj}}
\newcommand{\Perp}{\mathrm{perp}}
\newcommand{\proj}{\mathrm{proj}}
\newcommand{\Span}{\mathrm{span}}
\newcommand{\Null}{\mathrm{null}}
\newcommand{\Rank}{\mathrm{rank}}
\newcommand{\mat}[1]{\begin{bmatrix}#1\end{bmatrix}}
\renewcommand{\d}{\mathrm{d}}

%%%
% This is where the body of the document goes
%%%
\begin{document}
	\setheader{Math 281-2}{Homework 1}{Due Thursday, January 14}{}{}{}
	\begin{enumerate}
		\item Remember tangent planes and linear approximations?  Given a function $f(x,y)$, 
			we approximated it at the point $\vec p$ as $f(\vec p+\vec x)\approx L(\vec x) + f(\vec p)$,
			where $L$ was a \emph{linear function}.  To be more precise, $L(x_1,\ldots, x_n)=\sum \alpha_ix_i$
			for some choice of $\alpha_i$, and that choice of $\alpha_i$ happens to be the directional derivative
			of $f$ in the $\hat x,\hat y,$ etc\mbox{.} directions.
			\begin{enumerate}
				\item Let $f(x) = x^3$.  Find a linear approximation to $f$ at $x=2$.
					That is, find a linear function $L$ so that $f(2+x)\approx L(x) + f(2)$.
				\item Let $f(x,y) = -yx^2$.  Find a linear approximation to $f$ at $(x,y)=(2,3)$.
					That is, find a linear function $L$ so that $f((2,3)+(x,y))\approx L(x,y) + f(2,3)$.
				\item Let $f(x,y,z) = -yx^2+z^3$.  Find a linear approximation to $f$ at $(x,y,z)=(2,3,1)$.
				\item Now let's do something we've never done before.  Consider the vector field
					\[
						\vec f(x,y,z) = \mat{xy\\-z^2\\zx}=\mat{p(x,y,z)\\q(x,y,z)\\r(x,y,z)}.
					\]
				We'd like to find a linear approximation of $\vec f$ at the point $(x,y,z)=(2,3,1)$.
				The easiest way to do this is to find linear approximations for $p,q,$ and $r$, and 
				stick them in as the components of our linear approximation of $f$.  I'd like your answer
				to look like $\vec f((2,3,1)+(x,y,z)) = \vec L(x,y,z) + \vec f(2,3,1)$ where each component
				of $\vec L$ is a linear function.
				\item If a vector field is given by a linear function $\vec L$, must the curl of $\vec L$
					be zero?  Prove or give a counter example.
			\end{enumerate}
		\item In the definition of curl, we found the circulation around a rectangle and shrunk that rectangle
			to zero.  What happens if we tried another shape, like a circle?  Let's test this in the simple
			case of finding $\text{curl}_{\hat z}(\vec f)$ where $\vec f$ is linear.

			For this problem, we will assume
			\[
				\vec f(x,y,z) = \mat{A_x(x,y,z)\\B_y(x,y,z)\\D_z(x,y,z)} = \mat{\alpha_xx+\alpha_yy+\alpha_zz\\
				\beta_xx+\beta_yy+\beta_zz\\\delta_xx+\delta_yy+\delta_zz}
			\]
			\begin{enumerate}
				\item Find $\partial/\partial x$, $\partial/\partial y$, and $\partial/\partial z$ of $\vec f$.
				\item Parameterize $C$, a counter-clockwise oriented circle with radius $r$, centered at the origin 
					and lying in the $xy$-plane.
				\item Set up and evaluate an integral for the circulation of $\vec f$ around $C$.
				\item Evaluate $\displaystyle\lim_{r\to 0} \frac{\text{circulation around $C$}}{\text{area of $C$}}$.  Is
					this the same as the definition of curl using a rectangle?  Explain.
			\end{enumerate}

		\item Let $\vec f(x,y,z) = (-y,zx,x+z)$.
			\begin{enumerate}
				\item Find the curl of $\vec f$.
				\item Imagine $\vec f$ represents the force of the wind at
					every point in space.  If you place a tiny ball at $\vec a=(1,2,3)$,  
					will it start spinning?  What will be its axis of rotation?
				\item Let $H$ be the upper half of a sphere of radius $2$ centered
					at the origin (i.e., the part of the sphere with positive $z$ component). 
					Compute the circulation around $\partial H$ and the flux of $\nabla \times \vec f$
					through $H$.  How do they compare?
			\end{enumerate}
		\item Plot each of the following vector fields (your picture can be 2d).  
			From your picture, estimate divergence and curl, then compute
			the divergence and curl.
			\begin{enumerate}
				\item $\vec F(x,y,z) = (-y,x,0)$
				\item $\vec G(x,y,z) = (x,2y,0)$
				\item $\vec H(x,y,z) = (x,-y,0)$
				\item $\vec J(x,y,z) = (\frac{-y}{x^2+y^2}, \frac{x}{x^2+y^2}, 0)$
				\item A vector field is called \emph{closed} if it has zero curl.  The Poincar\'e
					lemma states that: \emph{If a vector field $\vec A:\R^3\to\R^3$ is closed
					on the interior of a sphere $\mathcal S$, then $\vec A$ is conservative 
					on the interior of $\mathcal S$.}  For each vector field above, explain 
					what the Poincar\'e lemma says about the existence of a potential function.
					If a global potential function exists (that is, there is some $f$
					so that $\nabla f = \vec A$ everywhere), write it down.  
					
					If only a local potential function exists at the point $\vec x$, find a
					sphere $\mathcal S_{\vec x}$ and a 
					potential $f_{\vec x}$ so that $\nabla f_{\vec x} = \vec A$ when restricted to the interior
					of $\mathcal S_{\vec x}$.  For simplicity, you may assume $\mathcal S_{\vec x}$ is centered at $\vec x$
					and just give its radius.
					
					Hint: pay special attention to where each vector field is defined and 
					where your potential function(s) are defined, and look at the previous homework set for if you need extra
					inspiration.

			\end{enumerate}

		\item Do one of the following:
			\begin{enumerate}
				\item type this homework assignment, or
				\item create a \LaTeX\ document that contains the sentence, ``Hi, my name is \emph{your name},
					and my favorite equation is $y=x^2+\frac{\alpha}{2}$,'' and turn in a printout
					with your homework. You are free to substitute your
					actual favorite equation in for $y=x^2+\frac{\alpha}{2}$.
			\end{enumerate}

	\end{enumerate}

\end{document}
