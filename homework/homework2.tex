\documentclass[letter]{article}
\usepackage{amsmath}
\usepackage{amsfonts}
\usepackage{amssymb}
\usepackage{ifthen}
\usepackage{fancyhdr}
\usepackage{enumitem}

%%%
% Set up the margins to use a fairly large area of the page
%%%
\oddsidemargin=.2in
\evensidemargin=.2in
\textwidth=6in
\topmargin=0in
\textheight=9.0in
\parskip=.07in
\parindent=0in
\pagestyle{fancy}

%%%
% Set up the header
%%%
\newcommand{\setheader}[6]{
	\lhead{{\sc #1}\\{\sc #2} ({\small \it \today})}
	\rhead{
		{\bf #3} 
		\ifthenelse{\equal{#4}{}}{}{(#4)}\\
		{\bf #5} 
		\ifthenelse{\equal{#6}{}}{}{(#6)}%
	}
}

%%%
% Set up some shortcut commands
%%%
\newcommand{\R}{\mathbb{R}}
\newcommand{\N}{\mathbb{N}}
\newcommand{\Z}{\mathbb{Z}}
\newcommand{\Proj}{\mathrm{proj}}
\newcommand{\Perp}{\mathrm{perp}}
\newcommand{\proj}{\mathrm{proj}}
\newcommand{\Span}{\mathrm{span}}
\newcommand{\Null}{\mathrm{null}}
\newcommand{\Rank}{\mathrm{rank}}
\newcommand{\mat}[1]{\begin{bmatrix}#1\end{bmatrix}}
\renewcommand{\d}{\mathrm{d}}

%%%
% This is where the body of the document goes
%%%
\begin{document}
	\setheader{Math 281-2}{Homework 2}{Due Thursday, January 28}{}{}{}
	\begin{enumerate}
		\item For each situation, set up a differential equation.  Explain what each function
			in your equation represents and explicitly specify what
			symbols represent constants.
		\begin{enumerate}
			\item Newton's law of cooling states that the change of temperature of an
				object is proportional to the difference between its temperature and
				the ambient temperature.  Model temperature with respect to time.
			\item Your pickle plant has a brine making machine.  Water with a salinity of
				10\% flows into a 100 liter tank at a rate of 0.2 liters/second.  The incoming
				liquid mixes fully in the tank and then the mixed liquid flows out at 
				0.2 liters/second.  Model the salinity with respect to time.
			\item An object is falling.  The second derivative with respect to time
				is proportional to (gravity $-F_R$) where $F_R$ is the force of air
				resistance.  Air resistance is proportional to the product of
				surface area and velocity.  Model velocity with respect to time.
			\item The logistic model of population growth states that the growth in population
				is proportional to the product of the population and the remaining resources.
				Further, resources decrease linearly with population.
				Model population with respect to time.
			\item Fibonacci's model for bunny populations stated that the growth of the population
				is equal to the number of adult bunnies.  Assuming it takes the
				bunnies 1 year to mature and that they never die, 
				model the population of bunnies with respect to time.
		\end{enumerate}
		\item A first order differential equation is called \emph{separable} if
			it can be written in the form $f(y)\frac{\d y}{\d x}= h(x)$.  Separable
			equations are easy to implicitly solve. By integrating both sides, we have
			$
				\int f(y)\frac{\d y}{\d x}\d x= \int h(x)\d x.
			$ Applying the chain rule to the left hand side gives us
			\[
				\int f(y)\d y = \int h(x)\d x.
			\]

			For each differential equation, state whether or not it is separable.
			If it is separable, find the set of implicit solutions.
			Further, find explicit solutions (i.e., solutions
			where $y$ is a function of $x$) to the initial value problem $y(x_0)=y_0$
			for every applicable $(x_0,y_0)$.  (For example, if the implicit
			solution were $x^2+y^2=K$, then the solution would be $y=\sqrt{(x_0^2+y_0^2)-x^2}$
			if $y_0> 0$ and $y=-\sqrt{(x_0^2+y_0^2)-x^2}$ if $y_0<0$.  If you encounter
			a similar situation, you must include both such solutions and describe which initial
			conditions give rise to which solution.)
			\begin{enumerate}
				\item $y' = \frac{1-2y}{x}$
				\item $y'=\frac{-xy}{x+1}$
				\item $y' = 3\sqrt[3]{y^2}$
				\item $xy'+y=y^2$
			\end{enumerate}


		\item Consider the differential equation $y'=Ky-y^3$ where $K$ is constant.
		\begin{enumerate}
			\item Draw a slope field for the differential equation for your choice of $K$.  Make sure it's
				detailed enough for you to see what's going on.
			\item Partition the set of initial conditions $(x_0,y_0)$ into sets that
				give rise to ``similar looking'' solutions and describe the types
				of solutions you get from each set.  Warning: your partitions may depend on $K$!
			\item Let $\vec x_0=(x_0,y_0)$ and $\vec x_0^*=(x_0^*,y_0^*)$ with $\|\vec x_0-\vec x_0^*\|<\varepsilon$.
				Let $y(x)$ and $y^*(x)$ be solutions to the respective initial value problems.
				The initial condition $\vec x_0$ is called \emph{bi-stable} if $|y(x)-y^*(x)|<C\varepsilon$
				for some fixed $C$.  Otherwise it is called \emph{unstable}.  
				It is called \emph{forward stable} if $|y(x)-y^*(x)|<C\varepsilon$ when $x\geq x_0$
				and \emph{backwards stable} if $|y(x)-y^*(x)|<C\varepsilon$ when $x\leq x_0$ (so stable implies
				both forward and backwards stable).

				Identify each region of of initial conditions as forward/backward/bi stable or unstable.
			\item  For $K=2$, use Euler's method with 5 steps
				to estimate $y(5)$ where $y$ is a solution
				to the initial value problem $(0,1)$. Plot your estimate.  
				What is going on?
			\item Repeat your calculation for $K=2$ using Euler's method with 50 steps.
				(You can use a computer for the 50 steps.)

			\item Use Euler's method and this differential equation (with an appropriate value for $K$) to approximate
				$\sqrt{7}$.
		\end{enumerate}

		\item Consider the system of differential equations
			\[
				\frac{\d x}{\d t} = -y\qquad
				\frac{\d y}{\d t} = x
			\]
			and the initial values $x(0)=1$ and $y(0)=0$.
		\begin{enumerate}
			\item Show that $(x,y) = (\cos t, \sin t)$ is a solution to this initial value problem.
			\item Euler's method would suggest you approximate a solution to this differential equation
				iteratively with the formulas 
				\[
					x_n=x_{n-1}-y_{n-1}\Delta t\qquad\text{and}\qquad
					y_n = y_{n-1}+x_{n-1}\Delta t.
				\]
				What happens when you do so?  Do you get a closed curve?
			\item Use the modified Euler's method
				\[
					x_n=x_{n-1}-y_{n-1}\Delta t\qquad\text{and}\qquad
					y_n = y_{n-1}+x_{n}\Delta t.
				\]
				What happens now?  Explain.
		\end{enumerate}

	\end{enumerate}

\end{document}
